\documentclass[a4paper]{scrartcl}
\usepackage[english]{babel}
\usepackage{amsthm}
\usepackage{listings}
\usepackage{xcolor}

\newtheorem{lem}{Lemma}


\lstset{
  frame=none,
  xleftmargin=2pt,
  stepnumber=1,
  numbers=left,
  numbersep=5pt,
  numberstyle=\ttfamily\tiny\color[gray]{0.3},
  belowcaptionskip=\bigskipamount,
  captionpos=b,
  escapeinside={*'}{'*},
  language=haskell,
  tabsize=2,
  emphstyle={\bf},
  commentstyle=\it,
  stringstyle=\mdseries\rmfamily,
  showspaces=false,
  keywordstyle=\bfseries\rmfamily,
  columns=flexible,
  basicstyle=\small\sffamily,
  showstringspaces=false,
  morecomment=[l]\%,
  moredelim=[is][\underbar]{_}{_},
}
\usepackage{array,amsmath}

\newcolumntype{L}{>$l<$}
\newcommand{\haskellCode}[1]{\lstinline|#1|}
\newcommand{\explanation}[1]{ \texttt{ \{ #1 \} }}

\begin{document}
\section*{Exercise 1}
\emph{Given the definition of the operator \lstinline|(++)|}
\begin{enumerate}
	\item \lstinline | [ ] ++ ys = ys|
    \item \lstinline | (x:xs) ++ ys = x : (xs ++ ys) |
\end{enumerate}
\subsection*{Property 1}
I will show by induction on xs that the following holds: \lstinline| xs ++ [ ] = xs|.
\begin{itemize}
\item \textbf{Base Case}: \lstinline| xs = [] |:
	$$
    \begin{array}{l}
    	\text{\lstinline | [] ++ [] |} \\
    	 \texttt{ \{apply definition of (++) case 1 \} }\\
        \text{\lstinline | [ ] |} \\
    \end{array}
	$$
\item \textbf{Step Case}: \lstinline| xs = (w:ws) |:
	$$
	\begin{array}{l}
	\text{\lstinline | (w:ws) ++ [] |} \\
    \texttt{\{ apply definition of (++) case 2 \}} \\
    \text{\lstinline | w : (ws ++ []) | } \\
    \texttt{\{ apply I.H. on the subexpression \lstinline | ws ++ [] | \}} \\
    \text{\lstinline | w : ws | $\equiv$ \lstinline | xs |} \\
    \end{array}
    $$
\end{itemize}
\subsection*{Property 2}
I will show by induction on xs that \lstinline | xs ++ (ys ++ zs) = (xs ++ ys) ++ zs|.
\begin{itemize}
	\item \textbf{Base Case}: \lstinline| xs = [] |.
	$$
	\begin{array}{L}
	\lstinline| [] ++ (ys ++ zs) | \\
	\texttt{ \{ apply definition of (++) case 1 \} }\\
	\lstinline| ys ++ zs | \\
	\texttt{ \{unapply definition of (++) case 1 to ys \}} \\
	\lstinline| ([] ++ ys) ++ zs | \\
	\end{array}
	$$
	\item \textbf{Step Case}: \lstinline| xs = w:ws |.
	$$
	\begin{array}{L}
	\lstinline| (w:ws) ++ (ys ++ zs) | \\
	\texttt{ \{ apply definition of (++) case 2 \} }\\
	\lstinline| w: (ws ++ (ys ++ zs) | \\
	\texttt{ \{ apply I.H. on \lstinline| ws ++ (ys ++ zs)|  \}} \\
	\lstinline| w: ((ws ++ ys) ++ zs) | \\
	\texttt{ \{unapply case 2 of (++)\} } \\
	\lstinline| (w: (ws ++ ys)) ++ zs | \\
	\texttt{ \{unapply case 2 of (++) to  \lstinline| (w: (ws ++ ys)) | \}} \\
	\lstinline| ((w:ws) ++ ys) ++ zs | \\
	\texttt{ \{substituting (w:ws) with xs \}}\\
	\lstinline | (xs ++ ys) ++ zs | 
	\end{array}
	$$
\end{itemize}

\section*{Exercise 2}
\emph{Show that \emph{\lstinline|exec(c ++ d) s= exec d (exec c s)|}, where \lstinline|exec| is the function  that executes
	the code consisting of sequences of PUSH n and ADD operations.
}
I will show by induction on \haskellCode{c} that the property holds:
\begin{itemize}
	\item \textbf{Base Case}: \haskellCode{c = [] } 
	$$
	\begin{array}{L}
	\haskellCode{exec ([] ++ d) s}\\
	\explanation{Application of (++)} \\
	\haskellCode{exec d s} \\
	\explanation{unapplying exec (case 1)} \\
	\haskellCode{exec d (exec [] s)}\\
	\end{array}
	$$
	\item \textbf{Step Case} \haskellCode{c = (w:ws)}, i.e. the list contains at least one element.
	Because the stack is composed of \haskellCode{Push n} and \haskellCode{Add} operations, I have to
	distinguish two cases:
	\begin{enumerate}
		\item \haskellCode{c = (Push n):ws}
			$$
			\begin{array}{L}
			\haskellCode{exec (((Push n):ws) ++ d) s}\\
			\explanation{Application of (++)} \\
			\haskellCode{exec ((Push n) : (ws++d)) s} \\
			\explanation{Application of exec} \\
			\haskellCode{exec (ws ++ d) (n : s)}\\
			\explanation{Application of I.H.} \\
			\haskellCode{exec d (exec ws (n:s))} \\
			\explanation{unapply inner exec} \\
			\haskellCode{exec d (exec (Push n:ws) s)}\\
			\end{array}
			$$
			\item \haskellCode{c = ADD:ws}
			$$
			\begin{array}{L}
			\haskellCode{exec ((ADD:ws) ++ d) s}\\
			\explanation{Application of (++)} \\
			\haskellCode{exec (ADD : (ws++d)) s} \\
			\explanation{Application of exec}\\
			We assume that the stack is composed by at least 2 elements, i.e. \haskellCode{s} = \haskellCode{m:n:s'}. \\
			The previous line is:\\
			\haskellCode{exec (ADD : (ws++d)) (m:n:s')}\\
			\haskellCode{exec (ws ++ d) ((m+n) : s')}\\
			\explanation{Application of I.H.} \\
			\haskellCode{exec d (exec ws ((m+n):s'))} \\
			\explanation{unapply inner exec} \\
			\haskellCode{exec d (exec (ADD:ws) m:n:s')}\\
			\end{array}
			$$
	\end{enumerate}
	

\end{itemize}

\subsection*{Exercise 3}
Given the type and instance declarations below, verify the functor laws for the Tree type, by induction of trees:	
\begin{lstlisting}
data Tree a = Leaf a | Node (Tree a) (Tree a)

instance Functor Tree where
--fmap :: (a -> b) -> Tree a -> Tree b
fmap g (Leaf x) = Leaf (g x)
fmap g (Node l r) = Node (fmap g l) (fmap g r)
\end{lstlisting}
\subsection*{First Law}
I have to show that \haskellCode{fmap id = id}. I have to distinguish 
two cases: the base case \haskellCode{Leaf x} and the step one \haskellCode{Node l r}
\begin{itemize}
	\item \textbf{Base case}
	$$
	\begin{array}{L}
		\haskellCode{fmap id (Leaf x)}= \\
		\explanation{apply fmap}
		\haskellCode{Leaf (id x)}=\\
		\explanation{apply id}
		\haskellCode{Leaf x}=\\
		\explanation{unapply id}
		\haskellCode{id (Leaf x)}\\
	\end{array} 
	$$
	\item \textbf{Step Case}
	$$
	\begin{array}{L}
		\haskellCode{fmap id (Node l r)} =\\
		\explanation{apply fmap}\\
		\haskellCode{Node (_fmap id l_) (_fmap id r_)} =\\
		\explanation{I.H. on trees l and r}\\
		\haskellCode{Node (_id l_) (_id r_)}=\\
		\explanation{apply id on l and r}\\
		\haskellCode{Node l r}=\\
		\explanation{unaply id}\\
		\haskellCode{id (Node l r)}\\
	\end{array}
	$$
\end{itemize}

\subsection*{Second Law}
I have to show that \haskellCode{fmap (g.h) = fmap g. (fmap h)}. I have to distinguish 
two cases: the base case \haskellCode{Leaf x} and the step one \haskellCode{Node l r}
\begin{itemize}
	\item \textbf{Base case} (\haskellCode{Leaf x})
	$$
	\begin{array}{L}
	\haskellCode{fmap (g.h) (Leaf x)}= \\
	\explanation{apply fmap}\\
	\haskellCode{Leaf ((g.h) x)}=\\
	\explanation{apply composition of g and h}\\
	\haskellCode{Leaf (g (h x))}=\\
	\explanation{unapply fmap g} \\
	\haskellCode{fmap g (Leaf (h x))} = \\
	\explanation{unapply fmap h} \\
	\haskellCode{(fmap g . fmap h) (Leaf x)}\\

	\end{array} 
	$$
	\item \textbf{Step Case}
	$$
	\begin{array}{L}
	\haskellCode{fmap (g.h) (Node l r)} =\\
	\explanation{apply fmap}\\
	\haskellCode{Node (_fmap (g.h) l_) (_fmap (g.h) r_)} =\\
	\explanation{I.H. on trees l and r}\\
	\haskellCode{Node (_fmap g . fmap h l_) (_fmap g . fmap h r_)}=\\
	\explanation{unapply fmap g}\\
	\haskellCode{fmap g (Node (_fmap h l_) (_fmap h r_))}=\\
	\explanation{unaply fmap h}\\
	\haskellCode{fmap g . fmap h (Node l r)}\\
	\end{array}
	$$
\end{itemize}

\subsection*{Exercise 4}
Verify the functor laws for the Maybe type.
I recall the Maybe instantiation of the functor class:

\begin{lstlisting}
data Maybe a = Nothing | Just a

instance Functor Maybe where
--fmap :: (a -> b) -> Tree a -> Tree b
fmap g Nothing = Nothing
fmap g (Just x) = Just (g x)
\end{lstlisting}
\subsection*{First Law}
I have to show that \haskellCode{fmap id = id}. I have to distinguish 
two cases: \haskellCode{Nothing} and \haskellCode{Just x} (There is no induction here!)
\begin{itemize}
	\item \textbf{Case A} (\haskellCode{Nothing})
	$$
	\begin{array}{L}
	\haskellCode{fmap id Nothing}= \\
	\explanation{apply fmap} \\
	\haskellCode{Nothing}=\\
	\end{array} 
	$$
	\item \textbf{Case B} (\haskellCode{Just x})
	$$
	\begin{array}{L}
	\haskellCode{_fmap id (Just x)_} =\\
	\explanation{apply fmap}\\
	\haskellCode{Just (_id x_)} =\\
	\explanation{apply id}\\
	\haskellCode{_Just x_}=\\
	\explanation{unapply id}\\
	\haskellCode{id (Just x)}=\\

	\end{array}
	$$
\end{itemize}

\subsection*{Second Law}
I have to show that \haskellCode{fmap (g.h) = fmap g. (fmap h)}. I have to distinguish 
two cases: \haskellCode{Nothing} and \haskellCode{Just x}:
\begin{itemize}
	\item \textbf{Case A} (\haskellCode{Nothing})
	$$
	\begin{array}{L}
	\haskellCode{fmap (g.h) Nothing}= \\
	\explanation{apply fmap} \\
	\haskellCode{Nothing}=\\
	\explanation{unapply fmap g}\\
	\haskellCode{fmap g Nothing} = \\
	\explanation{unapply fmap h}\\
	\haskellCode{(fmap g.fmap h) Nothing}
	\end{array} 
	$$
	\item \textbf{Case B} (\haskellCode{Just x})
		$$
		\begin{array}{L}
		\haskellCode{fmap (g.h) (Just x)}= \\
		\explanation{apply fmap}\\
		\haskellCode{Just ((g.h) x)}=\\
		\explanation{apply composition of g and h}\\
		\haskellCode{Just (g (h x))}=\\
		\explanation{unapply fmap g} \\
		\haskellCode{fmap g (Just (h x))} = \\
		\explanation{unapply fmap h} \\
		\haskellCode{(fmap g . fmap h) (Just x)}\\
		
		\end{array} 
		$$
\end{itemize}
\section*{Exercise 5}
Given the equation \haskellCode{comp' e c = comp e ++ c}, show how to construct the recursive definition
for \haskellCode{comp'} by induction on \haskellCode{e}.
Before trying to solve the exercise I will show a very trivial LEMMA, that simplifies the following proofs:
\begin{lem}
	\label{lem:trivial}
	\haskellCode{[x]++ys = x:ys}
\end{lem}

\begin{proof}
	$$
	\begin{array}{L}
		\haskellCode{_[x]_++ys} \\
		\explanation{Desugaring}\\
		\haskellCode{_(x:[]) ++ ys_}\\
		\explanation{Apply definition of (++) step case}\\
		\haskellCode{x : (_[] ++ ys_)}\\
		\explanation{Apply definition of (++)}\\
		\haskellCode{x:ys}
	\end{array}
	$$
\end{proof}
\begin{itemize}
	\item \textbf{Base Case}:
	$$
	\begin{array}{L}
	\haskellCode{comp' (Val n) c}\\
	\explanation{apply given equation}\\
	\haskellCode{comp (Val n) ++ c}\\
	\explanation{apply comp}\\
	\haskellCode{[Push n] ++ c}\\
	\explanation{Lemma \ref{lem:trivial}}\\
	\haskellCode{(Push n) : c}\\
	\end{array}
	$$	
	\item \textbf{Step Case}:
	$$
	\begin{array}{L}
	\haskellCode{comp' (Add n m) c}\\
	\explanation{apply given equation}\\
	\haskellCode{comp (Add n m) ++ c}\\
	\explanation{apply comp}\\
	\haskellCode{comp n ++ comp m ++ ([ADD] ++ c)}\\
	\explanation{Lemma \ref{lem:trivial}}\\
	\haskellCode{comp n ++ comp m ++ (ADD : c)}\\
	\explanation{unapply comp'}\\
	\haskellCode{comp n ++ (comp' m (ADD : c))}\\
	\explanation{unapply comp'}\\
	\haskellCode{comp' n (comp' m (ADD : c))}
	\end{array}
	$$
\end{itemize}

\end{document}